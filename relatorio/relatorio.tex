\documentclass{article}

\usepackage[width=14cm, left=3cm, top=2cm]{geometry}

\usepackage[brazil]{babel}
\usepackage[T1]{fontenc}
\usepackage[utf8]{inputenc}

\usepackage{amsmath}
\usepackage{amssymb}

\usepackage{minted}

\usepackage{hyperref}

\newcommand{\lang}{\texttt{J}}
\newcommand{\inlcode}{\mintinline[breaklines]{J}}

\title{Trabalho Computacional - Programação Não Linear\\
Picross e Ant Colony}
\author{Daniel Kiyoshi Hashimoto Vouzella de Andrade - 119025937}
\date{Novembro 2022}

\begin{document}
\maketitle

\section{Problema: Picross}

O problema escolhido foi a resolução de
um puzzle de \emph{picross}.
Também conhecido por vários outros nomes
(nonogram, griddler, hanjie, pic-a-pix,
paint by numbers, \dots),
o puzzle tem como objetivo colorir
uma grid \(n \times m\) com quadrados,
seguindo \(n + m\) dicas
(exatamente uma para cada linha e coluna).
Cada dica é uma lista de números
que indicam o tamanho dos blocos e
sua ordem da sua respectiva linha ou coluna,
e assim o tamanho da lista
indica a quantidade de blocos dessa linha/coluna.

\subsection{Formas conhecidas de resolução}
\begin{itemize}
    \item Depth First Search (Brute Force): \par
    \item Heuristic/Iterational: \par
    \item Integer Linear Programming: \par
\end{itemize}

\section{Modelagem}
\subsection{Método/Heurística Escolhida}
\subsection{Funcionamento do Algoritmo}
\subsection{Função Objetivo}
\subsection{Espaço de Busca}

\section{Implementação}
\subsection{Representação do Grafo}
\subsection{Construção dos Caminhos}
\subsection{Atualização do Feromônio}
\subsection{Condição de Parada}

\section{Resultados}

\section{Possíveis Melhorias}
\subsection{Função Objetivo}
\subsection{Espaço de Busca}

\section{Análise final}

\section{Bibliografia}

Sobre \lang:
\begin{itemize}
    \item Página inicial: \par
        \url{https://wiki.jsoftware.com/wiki/Main_Page}
    \item Dicionário de primitivas: \par
        \url{https://wiki.jsoftware.com/wiki/NuVoc}
\end{itemize}

Sobre \emph{Picross}:
\begin{itemize}
    \item History of Picross (Nonograms): \par
        \url{https://youtu.be/qJCPxyi5x5g}
    \item Solving Colored Nonograms: \par
        \url{https://run.unl.pt/bitstream/10362/2388/1/Mingote_2009.pdf}
    \item NP Completeness of Nonograms: \par
        \url{https://kelbybsandvick.github.io/pdf/NPProblemPaper.pdf}
    \item Complexity and solvability of Nonogram puzzles: \par
        \url{https://fse.studenttheses.ub.rug.nl/15287/1/Master_Educatie_2017_RAOosterman.pdf}
    \item kamilkhanlab/nonogram-ilp
        (Implementação de solver em GAMS): \par
        \url{https://github.com/kamilkhanlab/nonogram-ilp}
    \item Nonogram Wikipédia: \par
        \url{https://en.wikipedia.org/wiki/Nonogram}
\end{itemize}

\end{document}
