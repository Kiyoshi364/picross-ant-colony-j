\documentclass{article}

\usepackage[width=14cm, left=3cm, top=2cm]{geometry}

\usepackage[brazil]{babel}
\usepackage[T1]{fontenc}
\usepackage[utf8]{inputenc}

%\usepackage{fontspec}
%\setmainfont{Noto Serif}

\usepackage{amsmath}
\usepackage{amssymb}

\usepackage{multirow}

\usepackage{minted}

\usepackage{hyperref}
\hypersetup{
    colorlinks=true,
    breaklinks=true,
    linkcolor=purple,
    filecolor=cyan,
    urlcolor=blue,
    pdftitle={Picross e Ant Colony},
}
\urlstyle{same}

\usepackage{subcaption}
\usepackage{graphicx}
\graphicspath{{../images/}}

\newcommand{\apl}{\texttt{APL}}
\newcommand{\lang}{\texttt{J}}
\newcommand{\jver}{\texttt{J903}}
\newcommand{\inlcode}{\mintinline[breaklines]{J}}

\newcommand{\file}{\texttt}

\title{Trabalho Computacional - Programação Não Linear\\
Picross e Ant Colony}
\author{Daniel Kiyoshi Hashimoto Vouzella de Andrade - 119025937}
\date{Novembro 2022}

\begin{document}
\maketitle

\section{Problema: Picross}

O problema escolhido foi a resolução de
um puzzle de \emph{picross}.
Também conhecido por vários outros nomes
(nonogram, griddler, hanjie, pic-a-pix,
paint by numbers, \dots),
o puzzle tem como objetivo colorir
uma grid \(n \times m\) com quadrados,
seguindo \(n + m\) dicas
(exatamente uma para cada linha e coluna).
Por simplificação, só vamos considerar
puzzles de tamanho \(n \times n\).

Cada dica é uma lista de números
que indicam o tamanho dos blocos e
sua ordem da sua respectiva linha ou coluna,
e assim o tamanho da lista
indica a quantidade de blocos dessa linha/coluna.

A figura \ref{pg:5x5} mostra dois exemplos
de puzzles \(n \times n\) já resolvidos,
mas note que as dicas não estão incluídas e
a borda cinza não faz parte do puzzle.
As dicas são representadas por uma matriz
\(2 \times n\) de lista de números
(eles estão ``encaixotados''),
de forma que a lista \((0,i)\)
indica a ordem e tamanho dos blocos
(esquerda para direita)
da linha \(i\)
(contando de cima para baixo).
E, de forma similar,
a lista \((1,i)\)
indica a ordem e tamanho dos blocos
(cima para baixo)
da coluna \(i\)
(contando da esquerda para direita).
Em ambos os casos \(0 \le i < n\).
Caso não tenha nenhum bloco na linha/coluna,
aparece um \(0\) no lugar de uma lista vazia.

A seguir, estão as dicas dos puzzles
\ref{p:face}, \ref{p:heart} e \ref{p:random5};
a linha que está identada é o código em \lang{}
e abaixo o resultado:
\begin{minted}{j}
    NB. $ é usado para alterar as dimensões do array
    2 5 $ 1 ; 1 1 ; 1 ; 1 1 ; 1 ; 0 ; 1 1 ; 0 ; 1 1 ; 3
+-+---+-+---+-+
|1|1 1|1|1 1|1|
+-+---+-+---+-+
|0|1 1|0|1 1|3|
+-+---+-+---+-+
   2 5 $ 1 1 ; 5 ; 5 ; 3 ; 1 ; 2 ; 4 ; 4 ; 4 ; 2
+---+-+-+-+-+
|1 1|5|5|3|1|
+---+-+-+-+-+
|2  |4|4|4|2|
+---+-+-+-+-+
   2 5 $ 2 2 ; 1 1 ; 1 1 ; 1 1 ; 1 ; 2 1 ; 1 2 ; 1 ; 1 2 ; 1
+---+---+---+---+-+
|2 2|1 1|1 1|1 1|1|
+---+---+---+---+-+
|2 1|1 2|1  |1 2|1|
+---+---+---+---+-+
\end{minted}

\begin{figure}[h]
    \centering
    \begin{subfigure}{0.3\textwidth}
        \centering
        \includegraphics[width=0.9\linewidth]{face-border}
        \caption{Carinha Feliz de ASCII}
        \label{p:face}
    \end{subfigure}
    \begin{subfigure}{0.3\textwidth}
        \centering
        \includegraphics[width=0.9\linewidth]{heart-border}
        \caption{Coração}
        \label{p:heart}
    \end{subfigure}
    \begin{subfigure}{0.3\textwidth}
        \centering
        \includegraphics[width=0.9\linewidth]{random5-border}
        \caption{Aleatório}
        \label{p:random5}
    \end{subfigure}
    \caption{Exemplos de picross 5x5}
    \label{pg:5x5}
\end{figure}

Ambos os problemas de existência e unicidade de uma solução
(saber se existe uma solução e se ela é única)
de um puzzle arbitrário de picross são \emph{NP-completo}.

\subsection{Formas conhecidas de resolução}
\begin{itemize}
    \item Depth First Search (Brute Force): \par
        Tenta pintar um pixel,
        se nenhuma dica quebrar tente de novo.
        Se alguma dica quebrar, troque a cor e continue.
        Se der errado com a nova cor, ``volte'' e tente de novo.
    \item Heuristic/Iterational: \par
        A ideia é aplicar uma sequência de regras simples
        para determinar a cor de algum pixel
        várias e várias vezes.
        A técnica é semelhante a um humano resolvendo o puzzle.
        Alguns puzzles mais complexos
        não podem ser resolvidos dessa forma,
        é comum usar um DFS quando
        uma iteração não encontra nenhum pixel novo.
    \item Integer Linear Programming: \par
        O puzzle é modelado como
        uma sequência de restrições lineares
        com uma função \emph{dummy} de otimização.
\end{itemize}

\section{Modelagem}

O algoritmo escolhido foi o \emph{Ant Colony}.
O principal motivo para ele ter sido escolhido
foi ``formigas são legais!'',
na verdade, ele foi escolhido antes do problema em si.
O problema é muito bom para essa heurística,
pois é o problema é facilmente traduzido
em escolher um caminho sobre um grafo e
grafo é gerado bem simples.
\emph{Ant Colony} funciona com um simples loop,
no qual a cada iteração são executados os passos:
\begin{enumerate}
    \item Construir caminhos \par
        Nesse passo, são construídos \(ants\) caminhos e
        o melhor deles, de acordo com a \emph{função objetivo},
        é guardado.
        Cada caminho, deve ser escolhido aleatóriamente
        em função da \emph{distribuição do feromônio} e
        da \emph{heurística}.
        Geralmente são usados os parâmetros \(\alpha\) e \(\beta\)
        para equilibrar esses valores
        que formam os pesos dos caminhos.
    \item Atualizar o feromônio \par
        O melhor caminho da iteração \(c\) é usado
        para atualizar
        o feromônio \(\tau_e\) em cada aresta \(e\) do grafo
        usando a seguinte fórmula:
        \[
            \tau_e \leftarrow
                (1 - \rho) \; \tau_e
                + in(c, e) \; \frac{\rho \; \tau}{len(c)}
        \]
        onde \(\rho\) é a taxa de evaporação do feromônio
        (um parâmetro),
        \(\tau\) é a soma de feromônio em todas as atestas
        (uma constante),
        \(in(c, e)\) é uma função que retorna \(1\) se
        \(c\) passa pela aresta \(e\) e \(0\) caso contrário e
        \(len(c)\) são por quantas arestas \(c\) passa.
    \item Checar a condição de parada \par
        Existem duas condições mais comuns:
        checar se ``acabou o gás'',
        podendo ser o número de iterações ou tempo de execução;
        checar se achamos uma solução.
        Nem sempre é possível ou fácil usar a segunda,
        mas nesse caso é possível e relativamente fácil.
        Quando for decidido que se deve parar,
        retorna-se o melhor caminho encontrado até agora
        (não necessariamente dessa iteração).
\end{enumerate}

\subsection{Função Objetivo}
A \emph{função objetivo} deve impor
uma ordenação sobre os caminhos,
de forma que quanto mais próxima da solução
``melhor'' ela é.

\subsection{Heurística}
A \emph{heurística} é uma forma de aumentar a chance de escolher
um caminho mais provável de ser melhor.
Também deixa pouco provável
caminhos que aparentam ser ruins.
Idealmente ela vai apontar para caminhos que
minimizam/maximizam a \emph{função objetivo}.

\section{Implementação}

A linguagem escolhida para a implementação foi \lang{}:
uma linguagem orientada a arrays,
da mesma família que \apl{}.
Não foi usada nenhuma biblioteca para construir o algoritmo,
mas foi usada \inlcode{viewmat} para fazer as imagens
apresentadas nesse relatório.

\subsection{Representação do Grafo e das Dicas}
Para um puzzle \(n \times m\),
construímos um grafo com \(1 + n \cdot m\) vértices
numerados de \(-1 \le v < n \cdot m\) e
\(2 \cdot n \cdot m\) arestas nomeadas \((p, b)\),
com \(0 \le p < n \cdot m\) e \(b \in \{0, 1\}\).

\(-1\) é o vértice inicial do caminho,
só está alí por formalidades;
os outros \(0 \le v < n \cdot m\)
representam um pixel de coordenadas \((x(v), y(v))\) do puzzle.
A princípio, \(x\) e \(y\) podem ser quaisquer
funções que, em conjunto,
não repetem nem deixam sobrar nenhum pixel.
Por simplicidade, vamos usar
\(x(v) := \!\!\!\mod\!\!(v, m)\) e
\(y(v) := \left\lfloor\frac{v}{m}\right\rfloor\),
mas, pela natureza da linguagem escolhida,
isso vai ``aparecer'' como a ausência de uma função
que reordena uma possível solução no código.

A aresta \((p, b)\) sai do vértice \(p-1\) e
chega no vértice \(p\),
\(b\) indica a cor do pixel \(p\):
se \(b = 0\) é vazio/branco,
é preto caso contrário.
Na prática podemos ``jogar os vértices fora''
e só prestar atenção nas arestas,
isso torna grafo é tão simples
a ponto de não aparecer explicitamente na implementação.

A representação das dicas já foi descrita na primeira seção,
mas não foi dito que ela só funciona para puzzles quadrados
(todas as linhas de uma matriz devem ter o mesmo tamanho).
Não é difícil criar uma outra representação que suporte
puzzles retangulares, basta encaixotar cada lista,
mas não tinha pensado em suportar puzzles retangulares durante
a implementação.
A dica do puzzle \ref{p:face} ficaria assim:
\begin{minted}{j}
   (1 ; 1 1 ; 1 ; 1 1 ; 1) ; < 0 ; 1 1 ; 0 ; 1 1 ; 3
+---------------+---------------+
|+-+---+-+---+-+|+-+---+-+---+-+|
||1|1 1|1|1 1|1|||0|1 1|0|1 1|3||
|+-+---+-+---+-+|+-+---+-+---+-+|
+---------------+---------------+
\end{minted}

\subsection{Representação de um Caminho}
O caminho é representado por uma matriz de mesma forma
do puzzle \(n \times m\) com os valores em \(\{ 0, 1 \}\).
O elemento na posição \((i, j)\) da matriz que guarda o valor \(b\)
representa a aresta \((xy^{-1}(i, j), b)\),
onde \(xy^{-1}\) é a ``função inversa'' das funções \(x\) e \(y\),
ou seja, \(xy^{-1}(i, j) = v\) quando \(x(v) = i\) e \(y(v) = j\).
No nosso caso, \(xy^{-1}(i, j) := i \cdot m + j\).

Essa representação é muito interessante, pois
``a menos de uma função de reorganização''
(no nosso caso, a ``reorganização vazia'')
o caminho é uma possível representação direta da solução.
Veja a representação da solução do puzzle \ref{p:heart}:
\begin{minted}{j}
    NB. Isso é só para guardar numa variável e mostrar o resultado
    ] b =: 0 1 0 1 0 , 1 1 1 1 1 , 1 1 1 1 1 , 0 1 1 1 0 ,: 0 0 1 0 0
0 1 0 1 0
1 1 1 1 1
1 1 1 1 1
0 1 1 1 0
0 0 1 0 0
    NB. 'X' é preto e '.' é branco
    b { '.X'
.X.X.
XXXXX
XXXXX
.XXX.
..X..
\end{minted}

\subsection{Função Objetivo}
A \emph{função objetivo} escolhida transforma
o caminho em uma matriz de lista de tamanho dos blocos,
de forma que,
se essa fosse a solução retornaria uma matriz idêntica a dica.
Então, as listas são pareadas e elemento a elemento
(possívelmente estendendo a lista menor com \(0\)s)
são feitas subtrações seguidas de absoluto (para remover negativos).
Somamos todos os números de todas as listas.
A função que faz isso é \inlcode{flist}.

Um exemplo comparando as segundas linhas dos
puzzles \ref{p:heart} e \ref{p:random5}:
\begin{minted}{j}
    5 ; 1 1         NB. em representação de blocos
+-+---+
|5|1 1|
+-+---+
    5 0 ,:!.0 1 1   NB. estendendo a menor com 0s
5 0
1 1
    | 5 0 - 1 1     NB. subtraindo, seguido de absoluto
4 1
    +/ 4 1          NB. somando (falta as outras linhas e colunas)
5
\end{minted}

Com essa escolha de função,
procuramos minimizar a sua imagem.
Uma propriedade interessante dela é
que apenas uma solução vai ter o valor de \(0\),
permitindo parar mais cedo.

\subsection{Representação do Feromônio}
A lista de feromônios é representada por uma
matriz \(2 \times n \times m\),
no qual o elemento \((b, i, j)\) armazena a
quantidade de feromônio da aresta \((xy^{-1}(i, j), b)\),
usando o \(xy^{-1}\) descrito anteriormente.

Colocar o \(b\) na primeira dimensão
é vantajoso pela semântica da linguagem.
Essa representação permite pegar
uma matriz que representa um caminho \(c\)
e ``colar em cima'' sua matriz com elementos complementares
(\(x \longmapsto 1 - x\)),
montando as respostas da função \(in(c, e)\)
(uma ``máscara'')
necessárias para atualizar o feromônio.
Também facilita na geração de caminhos,
explicada mais a frente.

\subsection{Construção de Caminhos}
Não foi escolhida nenhuma heurística
para a construção dos caminhos,
o principal motivo é simplificar a implementação
dado que estou fazendo o trabalho sozinho e
não ter pensado em uma heurística fácil de implementar.

Dessa forma, não precisamos
dos parâmetros \(\alpha\) e \(\beta\),
e a escolha do caminho,
pelo método da roleta,
é influenciada apenas pelo feromônio.
Isso, junto a escolha do grafo,
faz com que a escolha da cor de um pixel
seja completamente independente
da escolha da cor de qualquer outro pixel.
O esperado é que seja necessário mais iterações
para que se ache um caminho bom.

\subsection{Condição de Parada}
A condição de parada é a combinação
de um contador de iterações
com uma verificação da imagem da \emph{função objetivo},
se algum deles for \(0\) podemos parar.

\lang{} possui uma função de ``exponenciação de funções'',
que aplica a função a quantidade de vezes que o ``expoente'' indica,
de forma que para um \(n\) inteiro:
\[
    f^n(x) := \begin{cases}
        f^{n+1}(f^{-1}(x))  &\quad,\; n < 0 \\
        x                   &\quad,\; n = 0 \\
        f^{n-1}(f(x))       &\quad,\; n > 0
    \end{cases}
\]
Não precisa se preocupar com o inverso da função.
O interessante é que está definida para \(n = \infty\),
aplica-se \(f\) até o resultado não se alterar,
efetivamente calculando um \emph{ponto fixo} da função.
Na implementação, o loop principal retorna
a mesma entrada caso uma das condições indique para parar,
permitindo usar essa ideia de \emph{ponto fixo}.

O número máximo de iterações poderia ser \(n\) (o expoente),
no lugar de \(\infty\),
mas deixá-lo como um parâmetro
permite uma \inlcode{main} mais simples.

\section{Resultados}

Na seção \ref{s:bib},
tem um link com instruções de instalação.
A versão usada nesse trabalho foi a \jver{},
mas não deve ter grandes problemas se a versão for diferente.

Vamos seguir assumindo que já estamos em um sessão
(\texttt{jconsole}, \texttt{jqt} e \texttt{jhs} são exemplos),
e guardamos o caminho
até o \file{main.ijs} na variável \inlcode{path}.
Então, chamamos o verbo \inlcode{load}
para carregar as funções que estão no arquivo \file{main.ijs}.
Por exemplo:
\begin{minted}{j}
    NB. ']' é só para mostrar o valor da variável
    ] path =: '~/path/to/main.ijs'
~/path/to/main.ijs
    load path
    NB. Note que: load '~/path/to/main.ijs'
    NB. também funciona
\end{minted}

\subsection{Como rodar}
A função \inlcode{main} recebe
uma lista encaixotada de \emph{parâmetros},
vamos guardar em \inlcode{p}, e
uma lista encaixotada de \emph{variáveis de loop},
vamos guardar em \inlcode{v}.

Os \emph{parâmetros} são, em ordem:
\emph{taxa de evaporação} \(\rho\),
\emph{função objetivo} \(f\) e
\emph{número de formigas por iteração} \(ants\).
O exemplo assume que \(\rho = 0.4\),
\(f\) é está otimizando para o puzzle \ref{p:face} e
\(ants = 100\).
\begin{minted}{j}
    face
0 0 0 1 0
0 1 0 0 1
0 0 0 0 1
0 1 0 0 1
0 0 0 1 0
    n2b face    NB. calcula as dicas a partir de uma solução
+-+---+-+---+-+
|1|1 1|1|1 1|1|
+-+---+-+---+-+
|0|1 1|0|1 1|3|
+-+---+-+---+-+
    ] p =: (<0.4)`((n2b face)&f)`(<100)
+---+---------------------------+---+
|0.4|+-+-----------------------+|100|
|   ||&|+-------------------+-+||   |
|   || ||+-+---------------+|f|||   |
|   || |||0|+-+---+-+---+-+|| |||   |
|   || ||| ||1|1 1|1|1 1|1||| |||   |
|   || ||| |+-+---+-+---+-+|| |||   |
|   || ||| ||0|1 1|0|1 1|3||| |||   |
|   || ||| |+-+---+-+---+-+|| |||   |
|   || ||+-+---------------+| |||   |
|   || |+-------------------+-+||   |
|   |+-+-----------------------+|   |
+---+---------------------------+---+
    NB. Essa caixa do meio é a representação da função objetivo
\end{minted}

As \emph{váriavei de loop} são, em ordem:
\emph{contador de iterações} \(n\)
que conta quantas iterações ainda pode seguir,
\emph{lista de feromônio} \(\tau\),
\emph{imagem do caminho \emph{dummy}} e
\emph{caminho \emph{dummy}},
um caminho todo em branco para
servir como um \emph{``placeholder''}.
Elas podem ser criadas
com uma função auxiliar \inlcode{init}.
O exemplo assume que \(n = 200\) e
que a \(\tau\) começa com \(1\) em cada aresta.
\begin{minted}{j}
    NB. não precisa ser a solução,
    NB. basta uma matriz com o tamanho igual
    ] fer =: 1 newfer face
1 1 1 1 1
1 1 1 1 1
1 1 1 1 1
1 1 1 1 1
1 1 1 1 1

1 1 1 1 1
1 1 1 1 1
1 1 1 1 1
1 1 1 1 1
1 1 1 1 1
   ] v =: 200 (p init) fer
+---+---------+--+---------+
|200|1 1 1 1 1|14|0 0 0 0 0|
|   |1 1 1 1 1|  |0 0 0 0 0|
|   |1 1 1 1 1|  |0 0 0 0 0|
|   |1 1 1 1 1|  |0 0 0 0 0|
|   |1 1 1 1 1|  |0 0 0 0 0|
|   |         |  |         |
|   |1 1 1 1 1|  |         |
|   |1 1 1 1 1|  |         |
|   |1 1 1 1 1|  |         |
|   |1 1 1 1 1|  |         |
|   |1 1 1 1 1|  |         |
+---+---------+--+---------+
\end{minted}

O algoritmo propriamente dito,
está implementado em \inlcode{loop},
mas a \inlcode{main} coleta mais informações e
é mais fácil de usar.

A função \inlcode{main} pode ser chamada com os argumentos
\inlcode{p} pela esquerda e \inlcode{v} pela direita.
Antes de chamá-la, pode-se alterar a semente inicial
se achar interessante.
\inlcode{main} retorna uma lista de caixas,
a primeira caixa contém a semente inicial
com o estado do gerador de números aleatórios
antes de rodar o algoritmo,
mas esse estado é gigantesco,
então não vai ser mostrado no exemplo;
a segunda o tempo de execução;
a terceira é uma variável interna que indica se deve parar,
já que ele retornou algo, deve ser sempre \(0\);
a quarta são as \emph{variáveis de loop}:
\(n\), \(\tau\), \(f(c^*)\), \(c^*\).
\begin{minted}{j}
    NB. Escolhendo uma semente (opcional)
    wseed 16807

    'rng time stop ret' =: p main v
    time ; stop ; <ret
+-------+-+------------------------------------------------------------+
|2.03215|0|+-+--------------------------------------------+-+---------+|
|       | ||0| 1.32328 0.929616 0.620616   1.4589 0.859697|4|1 1 0 1 0||
|       | || |0.784592  1.43173  1.20543  1.51852  1.61536| |1 1 1 1 1||
|       | || | 1.36438  1.78557 0.427282  1.19983  1.08449| |1 1 0 0 0||
|       | || |0.632997 0.864541 0.512813 0.708474  1.00561| |0 0 0 1 0||
|       | || |0.453377 0.409504  1.34556 0.847346  1.18167| |0 1 1 1 0||
|       | || |                                            | |         ||
|       | || |0.676724  1.07038  1.37938 0.541097   1.1403| |         ||
|       | || | 1.21541 0.568269 0.794566 0.481477 0.384644| |         ||
|       | || |0.635621 0.214428  1.57272 0.800168 0.915508| |         ||
|       | || |   1.367  1.13546  1.48719  1.29153 0.994391| |         ||
|       | || | 1.54662   1.5905 0.654442  1.15265 0.818328| |         ||
|       | |+-+--------------------------------------------+-+---------+|
+-------+-+------------------------------------------------------------+
    NB. Se não quiser guardar a saída em variáveis pode fazer:
    NB. }. p main v
    NB. para jogar fora o 'rng', ou
    NB. 1 3 { p main v
    NB. para escolher quais resultados quer ver
    NB. (nesse caso 'time' e 'ret')
\end{minted}

\subsection{Método}

Para cada puzzle e configuração de parâmetros escolhidos,
vai setar uma seed, nesse caso \(16807\),
(isso é para poder reproduzir com mais facilidade) e
vão ser executados 5 vezes,
assim a tabela irá mostrar
os tempos e imagens de cada execução.
Para cada tabela, vão aparecer 3 imagens:
a solução, a diferença e
um melhor caminho encontrado em todas as execuções.
A diferença tem duas cores extras: vermelho e verde.
Vermelho aponta um pixel colorido erroniamente,
enquanto verde aponta um pixel que não foi pintado.

O puzzle escolhido foi \ref{p:face}.
Rodar várias vezes, salvar as imagens e montar a tabela
são tarefas repetitivas e consomem um tempo considerável.
Não tive tempo sobrando para automatizá-las,
idealmente, teria tabelas de todos os puzzles.

Os 3 puzzles da figura \ref{pg:others},
forams gerados aleatóriamente
para cada um dos seguintes tamanhos:
\(10 \times 10\) (\ref{p:random10}),
\(25 \times 25\) (\ref{p:random25}) e
\(50 \times 50\) (\ref{p:random25}).
A ideia seria ter uma noção de o quão bem
ele se comporta com puzzles maiores.

\begin{figure}[h]
    \centering
    \begin{subfigure}{0.3\textwidth}
        \centering
        \includegraphics[width=0.9\linewidth]{random10-border}
        \caption{Puzzle \(10 \times 10\)}
        \label{p:random10}
    \end{subfigure}
    \begin{subfigure}{0.3\textwidth}
        \centering
        \includegraphics[width=0.9\linewidth]{random25-border}
        \caption{Puzzle \(25 \times 25\)}
        \label{p:random25}
    \end{subfigure}
    \begin{subfigure}{0.3\textwidth}
        \centering
        \includegraphics[width=0.9\linewidth]{random50-border}
        \caption{Puzzle \(50 \times 50\)}
        \label{p:random50}
    \end{subfigure}
    \caption{Puzzles maiores}
    \label{pg:others}
\end{figure}

As imagens estão
na pasta \file{images/face/params/},
onde \file{params} são os parâmetros escolhidos
para a execução.
Entretanto tive problemas em tirar as fotos das respostas,
então não tenho certeza se todas estão certas.

\subsection{Tabelas}

\begin{table}[h]
    \centering
    \begin{tabular}{|lc|cc|cc|cc|} \hline
        && \multicolumn{6}{|c|}{\bfseries Evaporação} \\\hline
        \multicolumn{2}{|c|}{\bfseries Iterações}
            & \multicolumn{2}{|c|}{$\rho = 0.05$}
            & \multicolumn{2}{|c|}{$\rho = 0.25$}
            & \multicolumn{2}{|c|}{$\rho = 0.45$} \\\hline
        \multicolumn{2}{|c|}{$ants = 2500$}
            & \textbf{Tempo} & \textbf{Imagem}
            & \textbf{Tempo} & \textbf{Imagem}
            & \textbf{Tempo} & \textbf{Imagem} \\\hline
        \multirow{5}{*}{100}
            & 1 & 24.17 s & 4 & 24.01 s & 4 & 24.18 & 4 \\
            & 2 & 24.13 s & 6 & 24.08 s & 6 & 24.27 & 4 \\
            & 3 & 24.02 s & 6 & 23.90 s & 6 & 24.14 & 6 \\
            & 4 & 24.07 s & 4 & 23.98 s & 6 & 24.18 & 6 \\
            & 5 & 24.16 s & 4 & 23.90 s & 4 & 24.16 & 6 \\\hline
        \multirow{5}{*}{200}
            & 1 & 47.71 s & 4 & 47.63 s & 4 & 47.94 & 4 \\
            & 2 & 47.69 s & 4 & 47.84 s & 4 & 47.89 & 4 \\
            & 3 & 47.80 s & 4 & 47.53 s & 4 & 46.51 & 6 \\
            & 4 & 47.85 s & 4 & 47.90 s & 4 & 47.79 & 4 \\
            & 5 & 47.87 s & 4 & 47.99 s & 6 & 47.82 & 6 \\\hline
        \multirow{5}{*}{400}
            & 1 & 94.91 s & 4 & 95.11 s & 4 & 95.80 & 4 \\
            & 2 & 94.79 s & 6 & 95.23 s & 6 & 95.71 & 6 \\
            & 3 & 94.83 s & 4 & 95.30 s & 4 & 95.81 & 4 \\
            & 4 & 95.07 s & 6 & 95.20 s & 4 & 96.00 & 4 \\
            & 5 & 95.21 s & 4 & 95.69 s & 6 & 95.82 & 6 \\\hline
    \end{tabular}
    \caption{Execuções do Puzzle \ref{p:face}}
    \label{te:face}
\end{table}

\begin{figure}
    \centering
    \begin{subfigure}{0.3\textwidth}
        \centering
        \includegraphics[width=0.9\linewidth]{face-border}
        \caption{Solução}
    \end{subfigure}
    \begin{subfigure}{0.3\textwidth}
        \centering
        \includegraphics[width=0.9\linewidth]{face/evap005-it100/run1-diff-border}
        \caption{Diferença}
    \end{subfigure}
    \begin{subfigure}{0.3\textwidth}
        \centering
        \includegraphics[width=0.9\linewidth]{face/evap005-it100/run1-border}
        \caption{Melhor caminho}
    \end{subfigure}
    \caption{Melhor resultado}
\end{figure}

\section{Possíveis Melhorias de Performance}

As sugestões de melhorias são para melhorar
a qualidade do algoritmo,
não necessariamente o tempo de execução.

\subsection{Representação do Grafo}
Atualmente a distribuição de feromônio no Grafo
não guarda de nenhuma forma ``da onde'' a formiga veio.
Talvez uma representação que ``se lembre melhor'' ajude.

\subsection{Função Objetivo}
\subsubsection{Quadrado em vez de Absoluto}
A \emph{função objetivo} atual usa absoluto,
talvez elevar ao quadrado dê mais valor
a candidatos mais próximos a resposta.

\subsubsection{Função mais inteligente}
A \emph{função objetivo} atual converte o caminho em uma dica,
uma representação que perde informação.
Talvez haja uma forma melhor de medir proximidade da solução.

\subsection{Geração de Caminhos}
Não é usada nenhuma heurística para auxiliar
a geração de caminhos.
Um simples filtro de caminhos que quebram a dica
deve ajudar bastante.

\section{Possíveis Melhorias de Funcionalidade}
\subsection{Permitir para diferentes linhas e colunas}
O código só funciona para puzzles quadrados,
não deve ser muito difícil ajeitá-lo
para aceitar puzzles com dimensões retangulares.

\subsection{Generalizar para mais cores}
Semelhante ao anterior, não deve ser muito difícil.

\subsection{Permitir solução parcial}
Isso é algo bem legal, pois permitiria
resolver um puzzle parcialmente resolvido
(talvez depois de uma série de iterações lógicas).
Provavelmente daria um pouco mais de trabalho.

\section{Análise final}
Achei que os caminhos ficaram bem próximos,
mas aumentando o número de iterações
não ajudava em chegar mais próximo da resposta.
Deu a impressão de que ficou preso em um mínimo local
próximo à solução.

Achei que \lang{} cobriu bem as necessidades desse trabalho,
mas como é o meu primeiro projeto maior com essa linguagem,
acabei demorando mais do que o esperado.

Tive problemas em tirar as fotos das respostas,
então não tenho certeza se todas estão certas.

Eu acho que seria legal transformar esse trabalho
em um paper ou algo assim,
mas acho que não tenho muita disponibilidade de tempo para isso.
Então deveria ser algo pequeno.

\section{Bibliografia} \label{s:bib}

Sobre \emph{Trabalho}:
\begin{itemize}
    \item Repositório: \par
    \url{https://github.com/Kiyoshi364/picross-ant-colony-j}
\end{itemize}

\noindent
Sobre \lang{}:
\begin{itemize}
    \item Página inicial: \par
        \url{https://wiki.jsoftware.com/wiki/Main_Page}
    \item Instruções de instalação: \par
        \url{https://wiki.jsoftware.com/wiki/System/Installation}
    \item Dicionário de primitivas: \par
        \url{https://wiki.jsoftware.com/wiki/NuVoc}
    \item J-playground (uma sessão online): \par
        \url{https://jsoftware.github.io/j-playground/bin/html/index.html}
\end{itemize}

\noindent
Sobre \emph{Picross}:
\begin{itemize}
    \item History of Picross (Nonograms): \par
        \url{https://youtu.be/qJCPxyi5x5g}
    \item Solving Colored Nonograms: \par
        \url{https://run.unl.pt/bitstream/10362/2388/1/Mingote_2009.pdf}
    \item NP Completeness of Nonograms: \par
        \url{https://kelbybsandvick.github.io/pdf/NPProblemPaper.pdf}
    \item Complexity and solvability of Nonogram puzzles: \par
        \url{https://fse.studenttheses.ub.rug.nl/15287/1/Master_Educatie_2017_RAOosterman.pdf}
    \item kamilkhanlab/nonogram-ilp
        (Implementação de solver em GAMS): \par
        \url{https://github.com/kamilkhanlab/nonogram-ilp}
    \item Nonogram Wikipédia: \par
        \url{https://en.wikipedia.org/wiki/Nonogram}
\end{itemize}

\end{document}
