\documentclass{article}

\usepackage[width=14cm, left=3cm, top=2cm]{geometry}

\usepackage[brazil]{babel}
\usepackage[T1]{fontenc}
\usepackage[utf8]{inputenc}

%\usepackage{fontspec}
%\setmainfont{Noto Serif}

\usepackage{amsmath}
\usepackage{amssymb}

\usepackage{minted}

\usepackage{hyperref}
\hypersetup{
    colorlinks=true,
    breaklinks=true,
    linkcolor=purple,
    filecolor=cyan,
    urlcolor=blue,
    pdftitle={Picross e Ant Colony},
}
\urlstyle{same}

\usepackage{subcaption}
\usepackage{graphicx}
\graphicspath{{../images/}}

\newcommand{\lang}{\texttt{J}}
\newcommand{\inlcode}{\mintinline[breaklines]{J}}

\title{Trabalho Computacional - Programação Não Linear\\
Picross e Ant Colony}
\author{Daniel Kiyoshi Hashimoto Vouzella de Andrade - 119025937}
\date{Novembro 2022}

\begin{document}
\maketitle

\section{Problema: Picross}

O problema escolhido foi a resolução de
um puzzle de \emph{picross}.
Também conhecido por vários outros nomes
(nonogram, griddler, hanjie, pic-a-pix,
paint by numbers, \dots),
o puzzle tem como objetivo colorir
uma grid \(n \times m\) com quadrados,
seguindo \(n + m\) dicas
(exatamente uma para cada linha e coluna).
Por simplificação, só vamos considerar
puzzles de tamanho \(n \times n\).

Cada dica é uma lista de números
que indicam o tamanho dos blocos e
sua ordem da sua respectiva linha ou coluna,
e assim o tamanho da lista
indica a quantidade de blocos dessa linha/coluna.

A figura \ref{pg:5x5} mostra dois exemplos
de puzzles já resolvidos,
mas note que as dicas não estão incluídas e
a borda cinza não faz parte do puzzle.
As dicas são representadas por uma matriz
\(2 \times n\) de lista de números
(eles estão ``encaixotados''),
de forma que a lista \((0,i)\)
indica a ordem e tamanho dos blocos
(esquerda para direita)
da linha \(i\)
(contando de cima para baixo).
E, de forma similar,
a lista \((1,i)\)
indica a ordem e tamanho dos blocos
(cima para baixo)
da coluna \(i\)
(contando da esquerda para direita).
Em ambos os casos \(0 \le i < n\).

A seguir, estão as dicas dos puzzles
\ref{p:face}, \ref{p:heart} e \ref{p:random5};
a linha que está identada é o código em \lang{}
e abaixo o resultado:
\begin{minted}{j}
    NB. $ é usado para alterar as dimensões do array
    2 5 $ 1 ; 1 1 ; 1 ; 1 1 ; 1 ; 0 ; 1 1 ; 0 ; 1 1 ; 3
+-+---+-+---+-+
|1|1 1|1|1 1|1|
+-+---+-+---+-+
|0|1 1|0|1 1|3|
+-+---+-+---+-+
   2 5 $ 1 1 ; 5 ; 5 ; 3 ; 1 ; 2 ; 4 ; 4 ; 4 ; 2
+---+-+-+-+-+
|1 1|5|5|3|1|
+---+-+-+-+-+
|2  |4|4|4|2|
+---+-+-+-+-+
   2 5 $ 2 2 ; 1 1 ; 1 1 ; 1 1 ; 1 ; 2 1 ; 1 2 ; 1 ; 1 2 ; 1
+---+---+---+---+-+
|2 2|1 1|1 1|1 1|1|
+---+---+---+---+-+
|2 1|1 2|1  |1 2|1|
+---+---+---+---+-+
\end{minted}

\begin{figure}[h]
    \centering
    \begin{subfigure}{0.3\textwidth}
        \centering
        \includegraphics[width=0.9\linewidth]{face-border}
        \caption{Carinha Feliz de ASCII}
        \label{p:face}
    \end{subfigure}
    \begin{subfigure}{0.3\textwidth}
        \centering
        \includegraphics[width=0.9\linewidth]{heart-border}
        \caption{Coração}
        \label{p:heart}
    \end{subfigure}
    \begin{subfigure}{0.3\textwidth}
        \centering
        \includegraphics[width=0.9\linewidth]{random5-border}
        \caption{Aleatório}
        \label{p:random5}
    \end{subfigure}
    \caption{Exemplos de picross 5x5}
    \label{pg:5x5}
\end{figure}

Ambos os problemas de existência e unicidade de uma solução
(saber se existe uma solução e se ela é única)
de um puzzle arbitrário de picross são \emph{NP-completo}.

\subsection{Formas conhecidas de resolução}
\begin{itemize}
    \item Depth First Search (Brute Force): \par
        Tenta pintar um pixel,
        se nenhuma dica quebrar tente de novo.
        Se alguma dica quebrar, troque a cor e continue.
        Se der errado com a nova cor, ``volte'' e tente de novo.
    \item Heuristic/Iterational: \par
        A ideia é aplicar uma sequência de regras simples
        para determinar a cor de algum pixel
        várias e várias vezes.
        A técnica é semelhante a um humano resolvendo o puzzle.
        Alguns puzzles mais complexos
        não podem ser resolvidos dessa forma,
        é comum usar um DFS quando
        uma iteração não encontra nenhum pixel novo.
    \item Integer Linear Programming: \par
        O puzzle é modelado como
        uma sequência de restrições lineares
        com uma função \emph{dummy} de otimização.
\end{itemize}

\section{Modelagem}

O algoritmo escolhido foi o \emph{Ant Colony}.
O principal motivo para ele ter sido escolhido
foi ``formigas são legais!'',
na verdade, ele foi escolhido antes do problema em si.
O problema é muito bom para essa heurística,
pois é o problema é facilmente traduzido
em escolher um caminho sobre um grafo e
grafo é gerado bem simples.
\emph{Ant Colony} funciona com um simples loop,
no qual a cada iteração são executados os passos:
\begin{enumerate}
    \item Construir caminhos \par
        Nesse passo, são construídos \(ants\) caminhos e
        o melhor deles, de acordo com a \emph{função objetivo},
        é guardado.
        Cada caminho, deve ser escolhido aleatóriamente
        em função da \emph{distribuição do feromônio} e
        da \emph{heurística}.
        Geralmente são usados os parâmetros \(\alpha\) e \(\beta\)
        para equilibrar esses valores
        que formam os pesos dos caminhos.
    \item Atualizar o feromônio \par
        O melhor caminho da iteração \(p\) é usado
        para atualizar
        o feromônio \(\tau_e\) em cada aresta \(e\) do grafo
        usando a seguinte fórmula:
        \[
            \tau_e \leftarrow
                (1 - \rho) \; \tau_e
                + in(p, e) \; \frac{\rho \; \tau}{len(p)}
        \]
        onde \(\rho\) é a taxa de evaporação do feromônio
        (um parâmetro),
        \(\tau\) é a soma de feromônio em todas as atestas
        (uma constante),
        \(in(p, e)\) é uma função que retorna \(1\) se
        \(p\) passa pela aresta \(e\) e \(0\) caso contrário e
        \(len(p)\) são por quantas arestas \(p\) passa.
    \item Checar a condição de parada \par
        Existem duas condições mais comuns:
        checar se ``acabou o gás'',
        podendo ser o número de iterações ou tempo de execução;
        checar se achamos uma solução.
        Nem sempre é possível ou fácil usar a segunda,
        mas nesse caso é possível e relativamente fácil.
        Quando for decidido que se deve parar,
        retorna-se o melhor caminho encontrado até agora
        (não necessariamente dessa iteração).
\end{enumerate}

\subsection{Função Objetivo}
A \emph{função objetivo} deve impor
uma ordenação sobre os caminhos,
de forma que quanto mais próxima da solução
``melhor'' ela é.

\subsection{Heurística}
A \emph{heurística} é uma forma de aumentar a chance de escolher
um caminho mais provável de ser melhor.
Também deixa pouco provável
caminhos que aparentam ser ruins.
Idealmente ela vai apontar para caminhos que
minimizam/maximizam a \emph{função objetivo}.

\section{Implementação}
\subsection{Representação do Grafo}
\subsection{Função Objetivo}
\subsection{Construção de Caminhos}
\subsection{Atualização do Feromônio}
\subsection{Condição de Parada}

\section{Resultados}

\section{Possíveis Melhorias}
\subsection{Função Objetivo}
\subsection{Espaço de Busca}

\section{Análise final}

\section{Bibliografia}

Sobre \lang{}:
\begin{itemize}
    \item Página inicial: \par
        \url{https://wiki.jsoftware.com/wiki/Main_Page}
    \item Dicionário de primitivas: \par
        \url{https://wiki.jsoftware.com/wiki/NuVoc}
\end{itemize}

Sobre \emph{Picross}:
\begin{itemize}
    \item History of Picross (Nonograms): \par
        \url{https://youtu.be/qJCPxyi5x5g}
    \item Solving Colored Nonograms: \par
        \url{https://run.unl.pt/bitstream/10362/2388/1/Mingote_2009.pdf}
    \item NP Completeness of Nonograms: \par
        \url{https://kelbybsandvick.github.io/pdf/NPProblemPaper.pdf}
    \item Complexity and solvability of Nonogram puzzles: \par
        \url{https://fse.studenttheses.ub.rug.nl/15287/1/Master_Educatie_2017_RAOosterman.pdf}
    \item kamilkhanlab/nonogram-ilp
        (Implementação de solver em GAMS): \par
        \url{https://github.com/kamilkhanlab/nonogram-ilp}
    \item Nonogram Wikipédia: \par
        \url{https://en.wikipedia.org/wiki/Nonogram}
\end{itemize}

\end{document}
